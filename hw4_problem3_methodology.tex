\documentclass[12pt]{article}
\usepackage{amsmath}
\usepackage{amssymb}
\usepackage{graphicx}
\usepackage{float}
\usepackage{geometry}
\usepackage{hyperref}
\usepackage{listings}
\usepackage{xcolor}

\geometry{margin=1in}

\title{High-Order Symplectic Integration of the Harmonic Oscillator\\
\large Using Yoshida Composition Methods}
\author{HW4 Problem 3}
\date{\today}

\begin{document}

\maketitle

\section{Problem Formulation}

\subsection{The Harmonic Oscillator}

The simple harmonic oscillator is described by the Hamiltonian:
\begin{equation}
H(q, p) = \frac{1}{2}(p^2 + q^2)
\label{eq:hamiltonian}
\end{equation}

where $q$ is the position and $p$ is the momentum (all physical constants set to unity). The equations of motion are:
\begin{align}
\frac{dq}{dt} &= \frac{\partial H}{\partial p} = p \label{eq:qdot}\\
\frac{dp}{dt} &= -\frac{\partial H}{\partial q} = -q \label{eq:pdot}
\end{align}

The exact solution is:
\begin{align}
q(t) &= q_0 \cos(t) + p_0 \sin(t) \\
p(t) &= p_0 \cos(t) - q_0 \sin(t)
\end{align}

with the energy conserved: $H(q(t), p(t)) = H(q_0, p_0) = E_0$ for all $t$.

\section{Symplectic Integration}

\subsection{Symplectic Structure}

Hamiltonian systems possess a symplectic structure that preserves phase space volume. A numerical integrator is \textbf{symplectic} if it preserves this structure exactly (up to machine precision).

\subsection{The Leapfrog Method}

The second-order leapfrog integrator splits the Hamiltonian operator into kinetic and potential parts and applies them in sequence:

\begin{equation}
S(h) = K\left(\frac{h}{2}\right) \circ D(h) \circ K\left(\frac{h}{2}\right)
\label{eq:leapfrog}
\end{equation}

where:
\begin{itemize}
    \item $K(\tau)$: ``Kick'' operator updates momentum: $p \rightarrow p - q\tau$ (uses force $F(q) = -q$)
    \item $D(\tau)$: ``Drift'' operator updates position: $q \rightarrow q + p\tau$
\end{itemize}

Explicitly, for a timestep $h$:
\begin{align}
p_{1/2} &= p_n - q_n \cdot \frac{h}{2} \quad \text{(half kick)} \label{eq:lf_kick1}\\
q_{n+1} &= q_n + p_{1/2} \cdot h \quad \text{(full drift)} \label{eq:lf_drift}\\
p_{n+1} &= p_{1/2} - q_{n+1} \cdot \frac{h}{2} \quad \text{(half kick)} \label{eq:lf_kick2}
\end{align}

This method is second-order accurate: $\mathcal{O}(h^2)$, and is symplectic.

\section{Yoshida Composition Method}

\subsection{Recursive Composition Formula}

The Yoshida method constructs higher even-order symplectic integrators by recursively composing lower-order ones. Given a symplectic integrator $S_{2k-2}(h)$ of order $2k-2$, we construct an order-$2k$ integrator:

\begin{equation}
S_{2k}(h) = S_{2k-2}(w_1 h) \circ S_{2k-2}(w_0 h) \circ S_{2k-2}(w_1 h)
\label{eq:yoshida_composition}
\end{equation}

The composition weights are determined analytically:
\begin{align}
w_1 &= \frac{1}{2 - 2^{1/(2k-1)}} \label{eq:w1}\\
w_0 &= 1 - 2w_1 \label{eq:w0}
\end{align}

Note that $w_0 < 0$ for $k \geq 2$, meaning we integrate ``backwards'' for part of the composition. This is essential for achieving higher-order error cancellation.

\subsection{Order Conditions}

The composition weights must satisfy:
\begin{enumerate}
    \item \textbf{Consistency}: $\sum_{i} w_i = 1$ (ensures the full timestep $h$ is covered)
    \item \textbf{Odd-order cancellation}: $\sum_{i} w_i^{2j+1} = 0$ for $j = 1, 2, \ldots, k-1$
\end{enumerate}

These conditions guarantee that odd-order error terms cancel, leaving only even-order errors.

\subsection{Construction of Yoshida Integrators}

Starting from the second-order leapfrog $S_2(h)$, we recursively apply equation~\eqref{eq:yoshida_composition}:

\subsubsection{Fourth-Order: $S_4(h)$}

For $k=2$:
\begin{align}
w_1 &= \frac{1}{2 - 2^{1/3}} \approx 1.351207191959657 \\
w_0 &= 1 - 2w_1 \approx -1.702414383919315
\end{align}

The composition $S_4(h) = S_2(w_1 h) \circ S_2(w_0 h) \circ S_2(w_1 h)$ requires \textbf{3 leapfrog substeps}.

\subsubsection{Sixth-Order: $S_6(h)$}

For $k=3$:
\begin{align}
w_1 &= \frac{1}{2 - 2^{1/5}} \approx 1.176451218628365 \\
w_0 &= 1 - 2w_1 \approx -1.352902437256730
\end{align}

The composition $S_6(h) = S_4(w_1 h) \circ S_4(w_0 h) \circ S_4(w_1 h)$ requires \textbf{9 leapfrog substeps} (since each $S_4$ uses 3 substeps).

\subsubsection{Eighth-Order: $S_8(h)$}

For $k=4$:
\begin{align}
w_1 &= \frac{1}{2 - 2^{1/7}} \approx 1.125968855870793 \\
w_0 &= 1 - 2w_1 \approx -1.251937711741586
\end{align}

The composition $S_8(h) = S_6(w_1 h) \circ S_6(w_0 h) \circ S_6(w_1 h)$ requires \textbf{27 leapfrog substeps} (since each $S_6$ uses 9 substeps).

\subsection{Implementation Details}

The recursive composition is implemented by computing all composition weights at initialization, then applying the base leapfrog integrator with scaled timesteps:

\begin{verbatim}
coefficients = get_yoshida_coefficients(order)
for each coefficient w in coefficients:
    (q, p) = leapfrog_step(q, p, w * dt)
\end{verbatim}

The number of substeps grows as $3^{(k-1)}$ for order $2k$.

\section{Comparison with Runge-Kutta 4}

The classical fourth-order Runge-Kutta (RK4) method is a non-symplectic integrator with the same formal order of accuracy ($\mathcal{O}(h^4)$) as Yoshida order 4. However, RK4 does not preserve the symplectic structure, leading to:
\begin{itemize}
    \item \textbf{Systematic energy drift}: For stable systems like the harmonic oscillator, RK4 exhibits a small but systematic energy bias per orbit that accumulates \emph{linearly} in time: $\Delta H \approx C \cdot t$
    \item \textbf{Phase error accumulation}: Even when energy appears conserved, phase space trajectories drift
\end{itemize}

In contrast, symplectic integrators like Yoshida methods exhibit \emph{bounded} energy oscillations without secular drift.

\section{Numerical Experiments}

\subsection{Initial Conditions and Parameters}

We integrate the harmonic oscillator with:
\begin{itemize}
    \item Initial conditions: $q_0 = 1.0$, $p_0 = 0.0$
    \item Initial energy: $E_0 = 0.5$
    \item \textbf{Timestep: $\Delta t = 0.1$}
    \item \textbf{Total steps: 10,000} (final time $t_f = 1000$, corresponding to $\sim$159 orbits)
    \item Methods compared: RK4, Yoshida order 4 (Y4), Yoshida order 6 (Y6), Yoshida order 8 (Y8)
\end{itemize}

All simulations use the same timestep $\Delta t = 0.1$ and integrate for 10,000 steps to enable direct comparison of accuracy.

\subsection{Energy Error Definition}

The relative energy error is computed as:
\begin{equation}
\frac{|\Delta H|}{|E_0|} = \frac{|H(q(t), p(t)) - E_0|}{|E_0|}
\end{equation}

This normalized dimensionless quantity allows fair comparison across different initial energies. For the harmonic oscillator, exact energy conservation means $\Delta H(t) = 0$ for all $t$.

\section{Results}

\subsection{RK4 Method Results}

The classical fourth-order Runge-Kutta method serves as our non-symplectic baseline.

\subsubsection{Implementation}

RK4 uses the standard four-stage explicit scheme:
\begin{align}
k_1 &= f(t_n, y_n) \\
k_2 &= f(t_n + \frac{h}{2}, y_n + \frac{h}{2}k_1) \\
k_3 &= f(t_n + \frac{h}{2}, y_n + \frac{h}{2}k_2) \\
k_4 &= f(t_n + h, y_n + hk_3) \\
y_{n+1} &= y_n + \frac{h}{6}(k_1 + 2k_2 + 2k_3 + k_4)
\end{align}

where $y = (q, p)^T$ and $f(t, y) = (p, -q)^T$ for the harmonic oscillator.

Each timestep requires \textbf{4 force evaluations}.

\begin{figure}[H]
\centering
\includegraphics[width=\textwidth]{rk4_results.png}
\caption{RK4 integration results: position (left) and energy error (right) versus time.}
\label{fig:rk4}
\end{figure}

\textbf{Observations from Figure~\ref{fig:rk4}:}
\begin{itemize}
    \item \textbf{Position plot (left)}: The oscillatory trajectory is visually indistinguishable from the exact solution over 159 orbits, demonstrating that RK4 accurately captures the periodic motion.
    \item \textbf{Energy error plot (right)}: Shows a clear \emph{monotonic linear drift} in the normalized relative energy error. The energy systematically decreases over time, reaching a maximum deviation of $1.39 \times 10^{-4}$ (0.014\% error) by $t=1000$.
    \item \textbf{Non-symplectic behavior}: The linear accumulation of error is characteristic of non-symplectic integrators. RK4 does not preserve the geometric structure of Hamiltonian systems, leading to secular drift rather than bounded oscillations.
    \item \textbf{Cumulative nature}: The error grows approximately as $\Delta H \propto t$, meaning longer integrations will continue to lose energy without bound.
\end{itemize}

\subsection{Yoshida Order 4 Results}

The fourth-order symplectic integrator uses three leapfrog substeps with composition weights.

\subsubsection{Implementation}

Composition: $S_4(h) = S_2(w_1 h) \circ S_2(w_0 h) \circ S_2(w_1 h)$ where:
\begin{align}
w_1 &= \frac{1}{2 - 2^{1/3}} \approx 1.351207191959657 \\
w_0 &= 1 - 2w_1 \approx -1.702414383919315
\end{align}

Each timestep requires \textbf{6 force evaluations} (3 substeps $\times$ 2 per leapfrog).

\begin{figure}[H]
\centering
\includegraphics[width=\textwidth]{yoshida4_results.png}
\caption{Yoshida-4 integration results: position (left) and energy error (right) versus time.}
\label{fig:y4}
\end{figure}

\textbf{Observations from Figure~\ref{fig:y4}:}
\begin{itemize}
    \item \textbf{Position plot (left)}: Maintains accurate periodic oscillations throughout the entire integration, identical in appearance to RK4.
    \item \textbf{Energy error plot (right)}: Exhibits \emph{bounded oscillations} around zero with amplitude $\sim 7.7 \times 10^{-6}$ (0.00077\% error). No secular drift is observed.
    \item \textbf{Symplectic preservation}: The bounded nature of the error demonstrates that Y4 preserves the symplectic structure. Energy oscillates but never systematically drifts away from the true value.
    \item \textbf{18$\times$ improvement over RK4}: Despite having the same formal order of accuracy ($\mathcal{O}(h^4)$), Y4 achieves significantly better energy conservation than RK4 due to structure preservation.
    \item \textbf{Computational cost}: Uses 1.5$\times$ more force evaluations than RK4 (6 vs 4), making this an excellent trade-off for long-term integration.
\end{itemize}

\subsection{Yoshida Order 6 Results}

The sixth-order symplectic integrator uses nine leapfrog substeps.

\subsubsection{Implementation}

Composition: $S_6(h) = S_4(w_1 h) \circ S_4(w_0 h) \circ S_4(w_1 h)$ where:
\begin{align}
w_1 &= \frac{1}{2 - 2^{1/5}} \approx 1.176451218628365 \\
w_0 &= 1 - 2w_1 \approx -1.352902437256730
\end{align}

Each timestep requires \textbf{18 force evaluations} (9 substeps $\times$ 2 per leapfrog).

\begin{figure}[H]
\centering
\includegraphics[width=\textwidth]{yoshida6_results.png}
\caption{Yoshida-6 integration results: position (left) and energy error (right) versus time.}
\label{fig:y6}
\end{figure}

\textbf{Observations from Figure~\ref{fig:y6}:}
\begin{itemize}
    \item \textbf{Position plot (left)}: Continues to show excellent agreement with the exact solution over the entire integration period.
    \item \textbf{Energy error plot (right)}: Bounded oscillations with dramatically reduced amplitude $\sim 9.2 \times 10^{-8}$ (0.000009\% error), nearly two orders of magnitude better than Y4.
    \item \textbf{Higher-order accuracy}: The sixth-order method shows the expected improvement in error magnitude, with error scaling as $\mathcal{O}(h^6)$.
    \item \textbf{84$\times$ improvement over Y4}: The increased computational cost (18 vs 6 force evaluations) is justified by the substantial gain in accuracy.
    \item \textbf{Long-term stability}: The bounded oscillations remain stable even after 159 orbits, demonstrating excellent phase space structure preservation.
\end{itemize}

\subsection{Yoshida Order 8 Results}

The eighth-order symplectic integrator uses twenty-seven leapfrog substeps.

\subsubsection{Implementation}

Composition: $S_8(h) = S_6(w_1 h) \circ S_6(w_0 h) \circ S_6(w_1 h)$ where:
\begin{align}
w_1 &= \frac{1}{2 - 2^{1/7}} \approx 1.125968855870793 \\
w_0 &= 1 - 2w_1 \approx -1.251937711741586
\end{align}

Each timestep requires \textbf{54 force evaluations} (27 substeps $\times$ 2 per leapfrog).

\begin{figure}[H]
\centering
\includegraphics[width=\textwidth]{yoshida8_results.png}
\caption{Yoshida-8 integration results: position (left) and energy error (right) versus time.}
\label{fig:y8}
\end{figure}

\textbf{Observations from Figure~\ref{fig:y8}:}
\begin{itemize}
    \item \textbf{Position plot (left)}: Maintains perfect periodic behavior indistinguishable from the exact solution.
    \item \textbf{Energy error plot (right)}: Bounded oscillations at the level of $\sim 7.2 \times 10^{-11}$ (0.0000000072\% error), approaching the limits of double-precision floating-point arithmetic.
    \item \textbf{Near machine precision}: The error magnitude is only about 5 orders of magnitude above machine epsilon ($\sim 2.2 \times 10^{-16}$ for double precision), indicating that numerical roundoff begins to dominate truncation error.
    \item \textbf{1,270$\times$ improvement over Y6}: Three orders of magnitude better than Y6, demonstrating the power of eighth-order accuracy.
    \item \textbf{Computational cost}: Requires 54 force evaluations per step (13.5$\times$ more than RK4), but delivers energy conservation 1.9 million times better than RK4.
    \item \textbf{Practical limit}: For this problem and timestep, Y8 represents near-optimal accuracy; further order increases would be limited by roundoff error rather than truncation error.
\end{itemize}

\subsection{Quantitative Energy Statistics}

Table~\ref{tab:energy_stats} summarizes the energy conservation statistics for each method over the entire integration period ($\Delta t = 0.1$, 10,000 steps):

\begin{table}[H]
\centering
\begin{tabular}{|l|c|c|c|c|}
\hline
\textbf{Method} & \textbf{Mean $|\Delta H/E_0|$} & \textbf{Std Dev $|\Delta H/E_0|$} & \textbf{Max $|\Delta H/E_0|$} & \textbf{Final $|\Delta H/E_0|$} \\
\hline
RK4       & $6.94 \times 10^{-5}$ & $4.00 \times 10^{-5}$ & $1.39 \times 10^{-4}$ & $1.39 \times 10^{-4}$ \\
Yoshida-4 & $3.83 \times 10^{-6}$ & $2.71 \times 10^{-6}$ & $7.66 \times 10^{-6}$ & $5.19 \times 10^{-6}$ \\
Yoshida-6 & $4.58 \times 10^{-8}$ & $3.24 \times 10^{-8}$ & $9.17 \times 10^{-8}$ & $6.27 \times 10^{-8}$ \\
Yoshida-8 & $3.61 \times 10^{-11}$ & $2.55 \times 10^{-11}$ & $7.22 \times 10^{-11}$ & $4.94 \times 10^{-11}$ \\
\hline
\end{tabular}
\caption{Normalized relative energy error statistics over $t \in [0, 1000]$ with $\Delta t = 0.1$ (10,000 steps). The normalized error $|\Delta H/E_0|$ is dimensionless and shows the fractional energy deviation.}
\label{tab:energy_stats}
\end{table}

\textbf{Key Observations:}
\begin{itemize}
    \item \textbf{RK4 exhibits systematic drift}: Final error equals maximum error ($1.39 \times 10^{-4}$), indicating monotonic energy loss without bounded oscillation.
    \item \textbf{Yoshida methods show bounded oscillations}: Final $\approx$ Mean for all Yoshida integrators, confirming energy oscillates around the true value.
    \item \textbf{Order scaling}: Y4 is 18$\times$ better than RK4; Y6 is 84$\times$ better than Y4; Y8 is 1,270$\times$ better than Y6.
    \item \textbf{Y8 approaches machine precision}: Relative errors below $10^{-10}$ demonstrate exceptional long-term stability.
\end{itemize}

\section{Discussion}

\subsection{Symplectic vs Non-Symplectic Integrators}

The results clearly demonstrate the fundamental difference between symplectic and non-symplectic integration schemes for Hamiltonian systems:

\textbf{Non-Symplectic Methods (RK4):}
\begin{itemize}
    \item \textbf{Structure}: Do not preserve the symplectic 2-form $dp \wedge dq$ that characterizes Hamiltonian phase space.
    \item \textbf{Energy behavior}: Exhibit systematic drift in conserved quantities. For our harmonic oscillator, RK4 shows linear energy loss: $\Delta H \approx -1.4 \times 10^{-7} \cdot t$.
    \item \textbf{Long-term dynamics}: Phase space volume is not preserved, leading to artificial dissipation or excitation. Over many orbits, trajectories drift away from the correct manifold.
    \item \textbf{Error accumulation}: Truncation errors in conserved quantities accumulate secularly, making long-term integration unreliable.
    \item \textbf{When to use}: Suitable for short-term integration, dissipative systems, or when high-order accuracy per step is more important than structure preservation.
\end{itemize}

\textbf{Symplectic Methods (Yoshida):}
\begin{itemize}
    \item \textbf{Structure}: Exactly preserve the symplectic structure at each timestep (up to machine precision).
    \item \textbf{Energy behavior}: Conserved quantities exhibit \emph{bounded oscillations} without secular drift. The "shadow Hamiltonian" theorem guarantees that symplectic integrators exactly conserve a nearby Hamiltonian $\tilde{H} = H + \mathcal{O}(h^p)$, where $p$ is the method order.
    \item \textbf{Long-term dynamics}: Phase space volume preservation (Liouville's theorem) is maintained. Trajectories remain on the correct energy surface for arbitrarily long times.
    \item \textbf{Error accumulation}: Errors oscillate but do not grow. This makes symplectic methods ideal for astronomical calculations, molecular dynamics, and other long-term Hamiltonian simulations.
    \item \textbf{When to use}: Essential for long-term integration of conservative systems, orbital mechanics, particle accelerators, and any application where preserving physical invariants is critical.
\end{itemize}

\textbf{Operator Splitting Philosophy:}

Symplectic integrators like Yoshida methods are based on \emph{operator splitting}, decomposing the Hamiltonian evolution into exactly solvable pieces:
\begin{equation}
e^{h\mathcal{L}} = e^{h(\mathcal{L}_T + \mathcal{L}_V)} \approx e^{w_1 h \mathcal{L}_T} e^{w_1 h \mathcal{L}_V} e^{w_0 h \mathcal{L}_T} e^{w_0 h \mathcal{L}_V} \cdots
\end{equation}

where $\mathcal{L}_T$ and $\mathcal{L}_V$ are the kinetic and potential Liouville operators. Each exponential is symplectic (being the exact flow of a Hamiltonian subsystem), so their composition is also symplectic. This is fundamentally different from RK4, which approximates the full derivative without respecting the Hamiltonian structure.

\subsection{Order of Accuracy and Computational Cost}

The computational cost scales with the number of force evaluations:
\begin{itemize}
    \item RK4: 4 evaluations per timestep
    \item Yoshida-4: 3 leapfrog substeps $\times$ 2 force evaluations = 6 evaluations
    \item Yoshida-6: 9 substeps $\times$ 2 = 18 evaluations
    \item Yoshida-8: 27 substeps $\times$ 2 = 54 evaluations
\end{itemize}

\textbf{Cost-Accuracy Trade-offs:}

\begin{table}[H]
\centering
\begin{tabular}{|l|c|c|c|}
\hline
\textbf{Method} & \textbf{Cost Ratio} & \textbf{Accuracy Gain} & \textbf{Efficiency} \\
                & (vs RK4)            & (vs RK4)                & (Gain/Cost) \\
\hline
RK4       & 1.0$\times$  & 1$\times$ (baseline)  & 1.0 \\
Yoshida-4 & 1.5$\times$  & 18$\times$            & 12.0 \\
Yoshida-6 & 4.5$\times$  & 1,500$\times$         & 333 \\
Yoshida-8 & 13.5$\times$ & 1,900,000$\times$     & 141,000 \\
\hline
\end{tabular}
\caption{Computational efficiency comparison: accuracy improvement per unit cost increase.}
\end{table}

\textbf{Analysis:}
\begin{itemize}
    \item \textbf{Yoshida-4}: Modest 50\% cost increase yields 18$\times$ better energy conservation. Excellent choice for most long-term integrations.
    \item \textbf{Yoshida-6}: The 4.5$\times$ cost increase is more than justified by the 1,500$\times$ accuracy improvement. Best overall efficiency ratio.
    \item \textbf{Yoshida-8}: Extreme accuracy at 13.5$\times$ the cost of RK4. Justified when near-machine-precision conservation is required or when timestep cannot be reduced.
    \item \textbf{Recursive cost}: The exponential growth ($3^{k-1}$) in substeps means that beyond order 8, computational cost becomes prohibitive for most applications.
\end{itemize}

\textbf{Timestep Selection:}

An alternative to increasing order is to reduce the timestep. For a method of order $p$, halving $\Delta t$ improves accuracy by $2^p$ but doubles the number of steps (and thus total cost). Comparing approaches:
\begin{itemize}
    \item To match Y8 accuracy, RK4 would need $\Delta t \approx 10^{-4}$ (1000$\times$ smaller), requiring 10 million steps vs Y8's 10,000 steps—a factor of 74$\times$ more expensive than Y8.
    \item High-order symplectic methods allow larger timesteps while maintaining accuracy and structure preservation, making them more efficient for long integrations.
\end{itemize}

\subsection{Practical Considerations}

\textbf{Advantages of Symplectic Methods:}
\begin{itemize}
    \item \textbf{Long-term stability}: Bounded errors make simulations reliable over millions of timesteps.
    \item \textbf{Physical fidelity}: Conserved quantities remain conserved, periodic orbits remain periodic.
    \item \textbf{Time reversibility}: Symplectic integrators are time-reversible, matching the physics.
    \item \textbf{No artificial damping}: Conservative systems remain conservative.
\end{itemize}

\textbf{Disadvantages of Symplectic Methods:}
\begin{itemize}
    \item \textbf{Hamiltonian-only}: Only applicable to systems that can be written in Hamiltonian form. Cannot handle velocity-dependent forces (e.g., magnetic fields in non-canonical coordinates) or dissipation directly.
    \item \textbf{Implicit schemes}: Higher-order symplectic methods often require implicit solves, though operator-splitting methods like Yoshida avoid this.
    \item \textbf{Negative timesteps}: Composition methods use $w_0 < 0$, which can cause issues in systems with singularities or discontinuities.
    \item \textbf{Setup complexity}: Requires separable Hamiltonian and implementation of operator splitting.
\end{itemize}

\textbf{Application Domains:}
\begin{itemize}
    \item \textbf{Celestial mechanics}: Planetary motion, satellite orbits, N-body gravitational systems.
    \item \textbf{Molecular dynamics}: Long-time simulation of molecular systems, protein folding.
    \item \textbf{Plasma physics}: Particle-in-cell simulations, beam dynamics in accelerators.
    \item \textbf{Classical field theory}: Lattice gauge theories, nonlinear wave equations.
\end{itemize}

\textbf{Application Domains:}
\begin{itemize}
    \item \textbf{Celestial mechanics}: Planetary motion, satellite orbits, N-body gravitational systems.
    \item \textbf{Molecular dynamics}: Long-time simulation of molecular systems, protein folding.
    \item \textbf{Plasma physics}: Particle-in-cell simulations, beam dynamics in accelerators.
    \item \textbf{Classical field theory}: Lattice gauge theories, nonlinear wave equations.
\end{itemize}

\section{Conclusions}

This study has demonstrated the superiority of symplectic integration methods over traditional non-symplectic schemes for long-term integration of Hamiltonian systems. The key findings are:

\begin{enumerate}
    \item \textbf{Structure preservation is crucial}: Symplectic integrators preserve the geometric structure of Hamiltonian phase space, leading to bounded energy errors. In contrast, RK4—despite being fourth-order accurate—exhibits systematic energy drift that accumulates linearly in time.
    
    \item \textbf{Yoshida composition delivers high-order symplectic methods}: By recursively composing the second-order leapfrog integrator with carefully chosen negative and positive timesteps, we achieved orders 4, 6, and 8 with relative energy errors of $7.7 \times 10^{-6}$, $9.2 \times 10^{-8}$, and $7.2 \times 10^{-11}$ respectively.
    
    \item \textbf{Dramatic accuracy improvements}: Using timestep $\Delta t = 0.1$ over 10,000 steps (159 orbits):
    \begin{itemize}
        \item Yoshida-4 is 18$\times$ more accurate than RK4
        \item Yoshida-6 is 1,500$\times$ more accurate than RK4
        \item Yoshida-8 is 1.9 million times more accurate than RK4
    \end{itemize}
    
    \item \textbf{Excellent cost-efficiency}: The computational cost increases moderately (1.5$\times$, 4.5$\times$, 13.5$\times$ for Y4, Y6, Y8 vs RK4) while accuracy improvements are dramatic. Yoshida-6 offers the best efficiency ratio with 333$\times$ better accuracy per unit cost.
    
    \item \textbf{Bounded vs unbounded errors}: The most important distinction is qualitative: RK4's errors grow without bound ($\Delta H \propto t$), making long-term integration unreliable. Yoshida methods exhibit bounded oscillations, remaining stable indefinitely.
    
    \item \textbf{Approaching machine precision}: Yoshida-8 achieves energy conservation at the $10^{-11}$ level, only 5 orders of magnitude above double-precision roundoff. This represents the practical limit of accuracy for this timestep.
    
    \item \textbf{Operator splitting philosophy}: The success of Yoshida methods stems from splitting the Hamiltonian evolution into exactly solvable kick and drift operations, each of which is symplectic. This differs fundamentally from RK4's Taylor-series approximation approach.
\end{enumerate}

\textbf{Practical Recommendations:}
\begin{itemize}
    \item For \textbf{moderate-accuracy} long-term integration: Use Yoshida-4. The 50\% cost increase over RK4 is negligible compared to the 18$\times$ accuracy gain and bounded error behavior.
    
    \item For \textbf{high-accuracy} applications: Use Yoshida-6. It provides the best balance of accuracy ($\sim 10^{-8}$ relative error) and computational cost.
    
    \item For \textbf{extreme-precision} needs: Use Yoshida-8, but be aware that roundoff errors may dominate. Consider using higher precision arithmetic if necessary.
    
    \item For \textbf{short-term integration} or \textbf{non-Hamiltonian systems}: RK4 remains a good choice due to its simplicity and efficiency.
\end{itemize}

\textbf{Broader Implications:}

The principle of geometric integration—designing numerical methods that preserve the geometric properties of differential equations—extends beyond symplectic methods. Similar structure-preserving techniques exist for:
\begin{itemize}
    \item Volume-preserving flows (divergence-free vector fields)
    \item Lie group integrators (for dynamics on manifolds)
    \item Variational integrators (discrete variational principles)
    \item Energy-preserving methods (discrete gradient methods)
\end{itemize}

For Hamiltonian systems requiring long-term integration, symplectic methods are not just preferable—they are essential. The Yoshida composition method provides a systematic, recursive way to achieve arbitrarily high order while maintaining symplecticity, making it a cornerstone technique in computational physics and geometric numerical integration.

\section*{References}

\begin{itemize}
    \item H. Yoshida, ``Construction of higher order symplectic integrators,'' \textit{Physics Letters A} \textbf{150}, 262--268 (1990).
    \item M. Suzuki, ``Fractal decomposition of exponential operators with applications to many-body theories and Monte Carlo simulations,'' \textit{Physics Letters A} \textbf{146}, 319--323 (1990).
    \item E. Hairer, C. Lubich, and G. Wanner, \textit{Geometric Numerical Integration: Structure-Preserving Algorithms for Ordinary Differential Equations}, 2nd ed., Springer (2006).
    \item Generation of values, plots, and equations was utilized with the assistance of generative AI using Github Copilot Pro
\end{itemize}

\end{document}
